\documentclass{article}

\usepackage{amsmath, amssymb, theorem}
\usepackage{hyperref}
\usepackage[margin=1in]{geometry}


\newcommand{\Y}{\mathbf{Y}{}}
\newcommand{\y}[2]{\mathbf{y}_{#1}^{#2}}
\newcommand{\one}[1]{\mathbf{1}\left(#1\right)}
\newcommand{\set}[1]{\left\{#1\right\}}

\begin{document}

\section{Preliminaries}

We are dealing with an array $\Y$ that has $|K|$ rows and $|N|$ columns, with $K$ is the set of functions and $N$ the set of genes.

Following the notation used in Exponential Random Graph Models, a sufficient statistic will be denoted by $g(\Y)$ and its correspondent change statistic that is defined switching $0\to1$ will be $\delta_{ij}^+ = g_{ij}(\Y^+) - g_{ij}(\Y^-)$. Without the + symbol, we represent the general version of it (so it can either apply for addition or deletion of a cell).

\section{Sufficient statistics}

\subsection{Single sibling-wide single function change}

We count how many instances we observe in which for a given function $k$ only either of two siblings had a change (regardless of whether it was gain/loss). 

\begin{align}
	g_i & = \sum_{\set{(n,m) \in N^2: n\neq m}}\left[\one{\y{in}{t}\neq\y{in}{t+1}}\one{\y{im}{t}=\y{im}{t+1}} + \one{\y{in}{t}=\y{in}{t+1}}\one{\y{im}{t}\neq\y{im}{t+1}}\right] \\
	\delta_i & =
\end{align}


\subsection{Number of gains and losses}

This adds one statistic equal to the number of gains or losses observed in the transition.

Gains

\begin{align}
g = & \sum_{i,n}\one{\y{in}{t} =0}\times\one{\y{in}{t+1}=1} \\
\delta = & (1 - \y{in}{t})(1 - 2\y{in}{t+1})
\end{align}

Losses

\begin{align}
g = & \sum_{i,n}\one{\y{in}{t} =1}\times\one{\y{in}{t+1}=0} \\
\delta = & \y{in}{t}(2\y{in}{t+1} - 1)
\end{align}


\end{document}