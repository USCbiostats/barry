\documentclass[11pt]{article}

\usepackage{amsmath, amssymb, theorem}
\usepackage{hyperref}
\usepackage[margin=1in]{geometry}
\usepackage{booktabs}
\usepackage{tabularx}

\newcommand{\Y}{\mathbf{Y}{}}
\newcommand{\y}[2]{\mathbf{y}_{#1}^{#2}}
\newcommand{\one}[1]{\mathbf{1}\left(#1\right)}
\newcommand{\set}[1]{\left\{#1\right\}}

\begin{document}

\section{Preliminaries}

We are dealing with an array $\Y$ that has $|K|$ rows and $|N|$ columns, with $K$ is the set of functions and $N$ the set of genes.

Following the notation used in Exponential Random Graph Models, a sufficient statistic will be denoted by $g(\Y)$ and its correspondent change statistic that is defined switching $0\to1$ will be $\delta_{ij}^+ = g_{ij}(\Y^+) - g_{ij}(\Y^-)$. Without the + symbol, we represent the general version of it (so it can either apply for addition or deletion of a cell).


\section{List of sufficient statistics}

\begin{itemize}
\item \textit{Gains/Loss counts} \textbf{(per function)} This function adds one sufficient statistic which equals to the number of functions that transitioned from 0 to 1 (or 0 to 1) across siblings.

Gains:

\begin{align}
g = & \sum_{i,n}\one{\y{in}{t} =0}\times\one{\y{in}{t+1}=1} \\
\delta = & (1 - \y{in}{t})(1 - 2\y{in}{t+1})
\end{align}

Losses:

\begin{align}
g = & \sum_{i,n}\one{\y{in}{t} =1}\times\one{\y{in}{t+1}=0} \\
\delta = & \y{in}{t}(2\y{in}{t+1} - 1)
\end{align}



\item \textit{Function swaps (negative correlation)} \textbf{(per pair of functions)} This adds one sufficient statistic that is equal to the number of times that two functions swapped values within the same gene, this is $(0,1) \to (1, 0)$.

\item \textit{Unique mutations} \textbf{(per pair of genes, per function)} This statistic equals to the number of times that, for each function x pair of genes, there was a single mutation, e.g.

$$
\left[\begin{array}{cc}
0 & 0 \\
0 & 1
\end{array}\right] \to
\left[\begin{array}{cc}
0 & 0 \\
\text{\textbf{1}} & 1
\end{array}\right]
$$

This could also be specified to work regardless of the function, this is, count how many times me see this behavior across functions.

\item \textit{Function correlation} \textbf{(per pair of functions)} Number of times during which two functions are either gained or lost simultaneusly.

\item \textit{Largest branch} \textbf{(per function)} This statistic equals to one if the longest branch is amongst those that transitioned

\item \textit{Largest branch (gain/loss)} \textbf{(per function)} This equals to one if the longest branch is amongst those that gain/loss a function.
\end{itemize}

\section{Sufficient statistics}

\subsection{Single sibling-wide single function change}

We count how many instances we observe in which for a given function $k$ only either of two siblings had a change (regardless of whether it was gain/loss). 

\begin{align}
	g_i & = \sum_{\set{(n,m) \in N^2: n\neq m}}\left[\one{\y{in}{t}\neq\y{in}{t+1}}\one{\y{im}{t}=\y{im}{t+1}} + \one{\y{in}{t}=\y{in}{t+1}}\one{\y{im}{t}\neq\y{im}{t+1}}\right] \\
	\delta_i & =
\end{align}



\end{document}